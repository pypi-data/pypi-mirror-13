\newcommand{\Title}{On Generalized Polynomial Chaos}
\newcommand{\TitleHeader}{}
\newcommand{\authorsHeader}{}
\newcommand{\authorsTitle}{Daniele Bigoni}
\newcommand{\courseTitle}{Documentation of the Spectral Toolbox}
\newcommand{\courseHeader}{}

%% Based on a TeXnicCenter-Template by Tino Weinkauf.
%%%%%%%%%%%%%%%%%%%%%%%%%%%%%%%%%%%%%%%%%%%%%%%%%%%%%%%%%%%%%

%%%%%%%%%%%%%%%%%%%%%%%%%%%%%%%%%%%%%%%%%%%%%%%%%%%%%%%%%%%%%
%% HEADER
%%%%%%%%%%%%%%%%%%%%%%%%%%%%%%%%%%%%%%%%%%%%%%%%%%%%%%%%%%%%%
\documentclass[a4paper,10pt]{article}
\usepackage[a4paper]{geometry}
% Alternative Options:
%	Paper Size: a4paper / a5paper / b5paper / letterpaper / legalpaper / executivepaper
% Duplex: oneside / twoside
% Base Font Size: 10pt / 11pt / 12pt
\usepackage{fancyhdr}

%% Language %%%%%%%%%%%%%%%%%%%%%%%%%%%%%%%%%%%%%%%%%%%%%%%%%
\usepackage[USenglish]{babel} %francais, polish, spanish, ...
\usepackage[T1]{fontenc}
\usepackage[ansinew]{inputenc}
\usepackage{subfig}

%% Packages for Graphics & Figures %%%%%%%%%%%%%%%%%%%%%%%%%%
\usepackage{graphicx} %%For loading graphic files
%\usepackage{subfig} %%Subfigures inside a figure
%\usepackage{tikz} %%Generate vector graphics from within LaTeX

%% Please note:
%% Images can be included using \includegraphics{filename}
%% resp. using the dialog in the Insert menu.
%% 
%% The mode "LaTeX => PDF" allows the following formats:
%%   .jpg  .png  .pdf  .mps
%% 
%% The modes "LaTeX => DVI", "LaTeX => PS" und "LaTeX => PS => PDF"
%% allow the following formats:
%%   .eps  .ps  .bmp  .pict  .pntg

\usepackage[usenames,dvipsnames]{xcolor}
\definecolor{light-gray}{gray}{0.95}

%% Math Packages %%%%%%%%%%%%%%%%%%%%%%%%%%%%%%%%%%%%%%%%%%%%
\usepackage{amsmath}
\usepackage{amsthm}
\usepackage{amsfonts}
\usepackage{verbatim}

\usepackage{mdframed}

\usepackage{fouridx}

% Statistical commands
\newcommand{\Exp}{\mathbf{E}}
\newcommand{\Var}{\mathbf{Var}}
\newcommand{\Cov}{\mathbf{Cov}}

% Number Representation
\newcommand{\complex}[3]{#1 #2 \mathit{i}\,#3}

% Set Representation
\newcommand{\field}[1]{\mathbb{#1}}

% Algebra
\DeclareMathOperator*{\tr}{trace} % Trace
\DeclareMathOperator*{\W}{W} % Wronski determinant
\DeclareMathOperator*{\diff}{d} % Differentiation

% Theorems
\newcommand{\myqed}{\begin{flushright}$\square$ \end{flushright}}
\newtheorem{thm}{Theorem}[section]
\newtheorem{mydef}{Definition}[section]
\newenvironment{myproof}{\noindent \textbf{Proof.}}{\myqed}
\newmdtheoremenv[outerlinewidth=1,leftmargin=5,%
	rightmargin=5,backgroundcolor=light-gray,%
	outerlinecolor=black,innertopmargin=5pt,%
	splittopskip=\topskip,skipbelow=\baselineskip,%
	skipabove=\baselineskip,ntheorem]{exa}%
	{Example}[section]


% Alter some LaTeX defaults for better treatment of figures:
% See p.105 of "TeX Unbound" for suggested values.
% See pp. 199-200 of Lamport's "LaTeX" book for details.
%   General parameters, for ALL pages:
\renewcommand{\topfraction}{0.9}	% max fraction of floats at top
\renewcommand{\bottomfraction}{0.8}	% max fraction of floats at bottom
%   Parameters for TEXT pages (not float pages):
\setcounter{topnumber}{2}
\setcounter{bottomnumber}{2}
\setcounter{totalnumber}{4}     % 2 may work better
\setcounter{dbltopnumber}{2}    % for 2-column pages
\renewcommand{\dbltopfraction}{0.9}	% fit big float above 2-col. text
\renewcommand{\textfraction}{0.07}	% allow minimal text w. figs
%   Parameters for FLOAT pages (not text pages):
\renewcommand{\floatpagefraction}{0.7}	% require fuller float pages
%N.B.: floatpagefraction MUST be less than topfraction !!
\renewcommand{\dblfloatpagefraction}{0.7}	% require fuller float pages

% Listing setting
\usepackage{listings}
\usepackage{color}
\lstset{
  language = matlab,                  % The programming language to be used
  frame = single,                     % Creates a frame around the listing
  % Options: single, double, shadowbow, t=top, b=bottom, r=right, l=left
  frameround=tfff,                    % Rounded corners of the frame
  showspaces = false,                 % Leave spaces blank?
  showstringspaces = false,           % Leave spaces in strings blank?
  breaklines = true,                  % Break lines that are too long?
  extendedchars = true,               % Includes danish characters?
  numbers = left,                     % Shows linenumbers?
  stepnumber = 1,                     % Linenumbers interval
  numberstyle = \footnotesize,        % The label size
  tabsize = 3,                        % Tabulator size (# of spaces)
  %xleftmargin = -20pt,               % Modification of left margin
  %xrightmargin = -20pt,              % Modification of right margin
  basicstyle = \ttfamily\small,       % The text style and size of the code
  commentstyle = \color[rgb]{0,0.5,0},% The style and color of the comments
  stringstyle = \color[rgb]{.5,0,.5},         % String colors
  keywordstyle = \bfseries\color[rgb]{0,0,1}, % Keyword colors
  keywords = {function,while,for,end,if,else,elseif}
  %%% Emphasize user-defined words:
  %emph = {[1] word1, word2 }, emphstyle = {[1] \bfseries\color[cmyk]{0.20,1.00,1.00,0.10}},
  %emph = {[2] wordA, wordB }, emphstyle = {[2] \color[cmyk]{0.20,1.00,1.00,0.10}},
}

%% Line Spacing %%%%%%%%%%%%%%%%%%%%%%%%%%%%%%%%%%%%%%%%%%%%%
%\usepackage{setspace}
%\singlespacing        %% 1-spacing (default)
%\onehalfspacing       %% 1,5-spacing
%\doublespacing        %% 2-spacing


%% Other Packages %%%%%%%%%%%%%%%%%%%%%%%%%%%%%%%%%%%%%%%%%%%
%\usepackage{a4wide} %%Smaller margins = more text per page.
%\usepackage{fancyhdr} %%Fancy headings
%\usepackage{longtable} %%For tables, that exceed one page


%%%%%%%%%%%%%%%%%%%%%%%%%%%%%%%%%%%%%%%%%%%%%%%%%%%%%%%%%%%%%
%% Remarks
%%%%%%%%%%%%%%%%%%%%%%%%%%%%%%%%%%%%%%%%%%%%%%%%%%%%%%%%%%%%%
%
% TODO:
% 1. Edit the used packages and their options (see above).
% 2. If you want, add a BibTeX-File to the project
%    (e.g., 'literature.bib').
% 3. Happy TeXing!
%
%%%%%%%%%%%%%%%%%%%%%%%%%%%%%%%%%%%%%%%%%%%%%%%%%%%%%%%%%%%%%

%%%%%%%%%%%%%%%%%%%%%%%%%%%%%%%%%%%%%%%%%%%%%%%%%%%%%%%%%%%%%
%% Options / Modifications
%%%%%%%%%%%%%%%%%%%%%%%%%%%%%%%%%%%%%%%%%%%%%%%%%%%%%%%%%%%%%

%
\section{\fpy command line options}
\label{sec:opts}

\fpy has the following command line syntax (run \fpy without arguments
to get up to date options!!!):
\begin{verbatim}
f2py [<options>] <fortran files> [[[only:]||[skip:]] <fortran functions> ]\
                 [: <fortran files> ...]
\end{verbatim}
where 
\begin{description}
\item[\texttt{<options>}] --- the following options are available:
  \begin{description}
  \item[\texttt{-f77}]  --- \texttt{<fortran files>} are in Fortran~77
    fixed format (default).
  \item[\texttt{-f90}]  --- \texttt{<fortran files>} are in
    Fortran~90/95 free format (default for signature files).
  \item[\texttt{-fix}] --- \texttt{<fortran files>} are in
    Fortran~90/95 fixed format.
  \item[\texttt{-h <filename>}] --- after scanning the
    \texttt{<fortran files>} write the signatures of Fortran routines
    to file \texttt{<filename>} and exit. If \texttt{<filename>}
    exists, \fpy quits without overwriting the file. Use
    \texttt{-{}-overwrite-signature} to overwrite.
  \item[\texttt{-m <modulename>}] --- specify the name of the module
    when scanning Fortran~77 codes for the first time. \fpy will
    generate Python C/API module source \texttt{<modulename>module.c}.
  \item[\texttt{-{}-lower/-{}-no-lower}]  --- lower/do not lower the cases
    when scanning the \texttt{<fortran files>}. Default when
    \texttt{-h} flag is specified/unspecified (that is for Fortran~77
    codes/signature files).
  \item[\texttt{-{}-short-latex}] --- use this flag when you want to
    include the generated LaTeX document to another LaTeX document.
  \item[\texttt{-{}-debug-capi}] --- create a very verbose C/API
    code. Useful for debbuging.
%  \item[\texttt{-{}-h-force}] --- if \texttt{-h <filename>} is used then
%    overwrite the file \texttt{<filename>} (if it exists) and continue
%    with constructing the C/API module source.
  \item[\texttt{-makefile <options>}] --- run \fpy without arguments
    for more information.
  \item[\texttt{-{}-use-libs}] --- see \texttt{-makefile}.
  \item[\texttt{-{}-overwrite-makefile}] --- overwrite existing
    \texttt{Makefile-<modulename>}.
  \item[\texttt{-v}] --- print \fpy version number and exit.
  \item[\texttt{-pyinc}] --- print Python include path and exit.
  \end{description}
\item[\texttt{<fortran files>}] --- are the paths to Fortran files or
  to signature files that will be scanned for \texttt{<fortran
    functions>} in order to determine their signatures.
\item[\texttt{<fortran functons>}] --- are the names of Fortran
  routines for which Python C/API wrapper functions will be generated.
  Default is all that are found in \texttt{<fortran files>}.
\item[\texttt{only:}/\texttt{skip:}] --- are flags for filtering
  in/out the names of fortran routines to be wrapped. Run \fpy without
  arguments for more information about the usage of these flags.
\end{description}


%%% Local Variables: 
%%% mode: latex
%%% TeX-master: "f2py2e"
%%% End: 
 %You need a file 'options.tex' for this
%% ==> TeXnicCenter supplies some possible option files
%% ==> with its templates (File | New from Template...).



%%%%%%%%%%%%%%%%%%%%%%%%%%%%%%%%%%%%%%%%%%%%%%%%%%%%%%%%%%%%%
%% DOCUMENT
%%%%%%%%%%%%%%%%%%%%%%%%%%%%%%%%%%%%%%%%%%%%%%%%%%%%%%%%%%%%%
\begin{document}

\pagestyle{empty} %No headings for the first pages.


%% Title Page %%%%%%%%%%%%%%%%%%%%%%%%%%%%%%%%%%%%%%%%%%%%%%%
%% ==> Write your text here or include other files.

%% The simple version:
\title{\course \\ \Title}
\author{\authorsTitle}

\begin{titlepage}
\begin{center}
\includegraphics[height=1.00in]{tex_dtu_logo}\\[1.5cm]
\textsc{\LARGE Technical University of Denmark}\\[3.0cm]
\textsc{\Large \Title}\\[1.0cm]
\textsc{\large \courseTitle}\\[3.0cm]
\begin{flushleft} \large
\emph{Author:}\\
\authorsTitle \\

\end{flushleft}
\vfill
% Bottom of the page
{\large \today}
%\date{} %%If commented, the current date is used.
\end{center}
\end{titlepage}

%% The nice version:
%\input{titlepage} %%You need a file 'titlepage.tex' for this.
%% ==> TeXnicCenter supplies a possible titlepage file
%% ==> with its templates (File | New from Template...).


%% Inhaltsverzeichnis %%%%%%%%%%%%%%%%%%%%%%%%%%%%%%%%%%%%%%%
\tableofcontents %Table of contents
\cleardoublepage %The first chapter should start on an odd page.

\pagestyle{fancy} %Now display headings: headings / fancy / ...



%% Chapters %%%%%%%%%%%%%%%%%%%%%%%%%%%%%%%%%%%%%%%%%%%%%%%%%
%% ==> Write your text here or include other files.

%
\section{Introduction}
\label{sec:intro}

\fpy is a command line tool that generates Python C/API modules for
interfacing Fortran~77/90/95 codes and Fortran~90/95 modules from
Python.  In general, using \fpy an
interface is produced in three steps:
\begin{itemize}
\item[(i)] \fpy scans Fortran sources and creates the so-called
  \emph{signature} file; the signature file contains the signatures of
  Fortran routines; the signatures are given in the free format of the
  Fortran~90/95 language specification. Latest version of \fpy
  generates also a make file for building shared module.  
  About currently supported compilers see the \fpy home page
\item[(ii)] Optionally, the signature files can be modified manually
  in order to dictate how the Fortran routines should be called or
  seemed from the Python environment.
\item[(iii)] \fpy reads the signature files and generates Python C/API
  modules that can be compiled and imported to Python code. In
  addition, a LaTeX document is generated that contains the
  documentation of wrapped functions.
\end{itemize}
(Note that if you are satisfied with the default signature that \fpy
generates in step (i), all three steps can be covered with just
one call to \fpy --- by not specifying `\texttt{-h}' flag).
Latest versions of \fpy support so-called \fpy directive that allows
inserting various information about wrapping directly to Fortran
source code as comments (\texttt{<comment char>f2py  <signature statement>}).

The following diagram illustrates the usage of the tool:
\begin{verbatim}
! Fortran file foo.f:
      subroutine foo(a)
      integer a
      a = a + 5
      end
\end{verbatim}
\begin{verbatim}
! Fortran file bar.f:
      function bar(a,b)
      integer a,b,bar
      bar = a + b
      end
\end{verbatim}
\begin{itemize}
\item[(i)] \shell{\fpy foo.f bar.f -m foobar -h foobar.pyf}
\end{itemize}
\begin{verbatim}
!%f90
! Signature file: foobar.pyf
python module foobar ! in
    interface  ! in :foobar
        subroutine foo(a) ! in :foobar:foo.f
            integer intent(inout) :: a
        end subroutine foo
        function bar(a,b) ! in :foobar:bar.f
            integer :: a
            integer :: b
            integer :: bar
        end function bar
    end interface
end python module foobar
\end{verbatim}
\begin{itemize}
\item[(ii)] Edit the signature file (here I made \texttt{foo}s
  argument \texttt{a} to be \texttt{intent(inout)}, see
  Sec.~\ref{sec:attributes}).
\item[(iii)] \shell{\fpy foobar.pyf}
\end{itemize}
\begin{verbatim}
/* Python C/API module: foobarmodule.c */
...
\end{verbatim}
\begin{itemize}
\item[(iv)] \shell{make -f Makefile-foobar}
%\shell{gcc -shared -I/usr/include/python1.5/ foobarmodule.c\bs\\
%foo.f bar.f -o foobarmodule.so}
\end{itemize}
\begin{verbatim}
Python shared module: foobarmodule.so
\end{verbatim}
\begin{itemize}
\item[(v)] Usage in Python:
\end{itemize}
\vspace*{-4ex}
\begin{verbatim}
>>> import foobar
>>> print foobar.__doc__
This module 'foobar' is auto-generated with f2py (version:1.174).
The following functions are available:
  foo(a)
  bar = bar(a,b)
.
>>> print foobar.bar(2,3)
5
>>> from Numeric import *
>>> a = array(3)
>>> print a,foobar.foo(a),a
3 None 8
\end{verbatim}
Information about how to call \fpy (steps (i) and (iii)) can be
obtained by executing\\
\shell{\fpy}\\
This will print the usage instructions.
 Step (iv) is system dependent
(compiler and the locations of the header files \texttt{Python.h} and
\texttt{arrayobject.h}), and so you must know how to compile a shared
module for Python in you system.

The next Section describes the step (ii) in more detail in order to
explain how you can influence to the process of interface generation
so that the users can enjoy more writing Python programs using your
wrappers that call Fortran routines.  Step (v) is covered in
Sec.~\ref{sec:notes}.


\subsection{Features}
\label{sec:features}

\fpy has the following features:
\begin{enumerate}
\item \fpy scans real Fortran codes and produces the signature files.
  The syntax of the signature files is borrowed from the Fortran~90/95
  language specification with some extensions.
\item \fpy uses the signature files to produce the wrappers for
  Fortran~77 routines and their \texttt{COMMON} blocks.
\item For \texttt{external} arguments \fpy constructs a very flexible
  call-back mechanism so that Python functions can be called from
  Fortran.
\item You can pass in almost arbitrary Python objects to wrapper
  functions.  If needed, \fpy takes care of type-casting and
  non-contiguous arrays.
\item You can modify the signature files so that \fpy will generate
  wrapper functions with desired signatures.  \texttt{depend()}
  attribute is introduced to control the initialization order of the
  variables. \fpy introduces \texttt{intent(hide)} attribute to remove
  the particular argument from the argument list of the wrapper
  function.  In addition, \texttt{optional} and \texttt{required}
  attributes are introduced and employed.
\item \fpy supports almost all standard Fortran~77/90/95 constructs
  and understands all basic Fortran types, including
  (multi-dimensional, complex) arrays and character strings with
  adjustable and assumed sizes/lengths.
\item \fpy generates a LaTeX document containing the
  documentations of the wrapped functions (argument types, dimensions,
  etc). The user can easily add some human readable text to the
  documentation by inserting \texttt{note(<LaTeX text>)} attribute to
  the definition of routine signatures.
\item \fpy generates a GNU make file that can be used for building
  shared modules calling Fortran functions.
\item \fpy supports wrapping Fortran 90/95 module routines.
\end{enumerate}

%%% Local Variables: 
%%% mode: latex
%%% TeX-master: "f2py2e"
%%% End: 
 %You need a file 'intro.tex' for this.


%%%%%%%%%%%%%%%%%%%%%%%%%%%%%%%%%%%%%%%%%%%%%%%%%%%%%%%%%%%%%
%% ==> Some hints are following:

\section{Introduction}
Why this paper?
Why Spectral Methods?
Why a Spectral Toolbox?
Why Python?

\section{Orthogonal Polynomials}
Orthogonal polynomials can be used in order to approximate functions in space. There exist infinite sets of orthogonal polynomials, however some of them have been studied extensively due to their simplicity and performances.\\

The Spectral Toolbox includes several polynomials for bounded as well as for unbounded domains. The available polynomials will be presented in the following. More information about orthogonal polynomials can be found in literature (e.g. \cite{shen_recent_2009}).

\subsection{Jacobi Polynomials}

\subsubsection{Legendre Polynomials}

\subsubsection{Chebyshev Polynomials}

\subsection{Hermite Polynomials}

Hermite polynomials span the interval $I:=(-\infty,\infty)$.

\subsubsection{Hermite Physicists' Polynomials}
The Hermite Physicists Polynomials denoted by $H_n(x)$ are eigenfunctions of the Sturm-Liouville problem:
\begin{equation}
e^{x^2}\left( e^{-x^2} H_n'(x) \right)' + \lambda_n H_n(x) = 0 , \qquad \forall x \in I:=(-\infty,\infty)
\end{equation}

\begin{itemize}
\item \textbf{Recurrence relation} 
	\begin{equation}
		\begin{cases}
		H_0(x) = 1 \\
		H_1(x) = 2x \\
		H_{n+1}(x) = 2xH_n(x) - 2nH_{n-1}(x)
		\end{cases}
	\end{equation}
\item \textbf{Derivatives} 
	\begin{equation}
		\begin{cases}
		H_n^{(k)}(x) = 2nH_{n-1}^{(k-1)}(x) \\
		H_n^{(0)}(x) = H_n(x) \\
		H_0^{(k)}(x) = 0 \qquad \text{for $k>0$}
		\end{cases}
	\end{equation}
\item \textbf{Orthogonality} 
	\begin{align}
		w(x) &= e^{-x^2} \\
		\gamma_n &= \sqrt{\pi} 2^n n!
	\end{align}
\item \textbf{Gauss Quadrature points and weights}\\
	The Gauss points $\lbrace x_j \rbrace_{j=0}^N$ corresponding to $H_{N+1}(x)$ can be obtained using the Golub-Welsh algorithm \cite{press_numerical_2007} where:
	\begin{equation}\label{eq:OrthPoly:HermitePolyPhy:GQx}
	a_j = 0 \qquad b_j = \frac{j}{2}
	\end{equation}
	The Gauss weights are obtained by:
	\begin{align}
	w_j = \frac{\lambda_N}{\lambda_{N-1}} \frac{(H_N(x),H_N(x))}{H_N(x_j)H_{N+1}'(x_j)} = \frac{\gamma_N}{(N+1)H_N^2(x_j)}
	\end{align}
\end{itemize}

\subsubsection{Hermite Functions}
Hermite Functions are used because of their better behavior respect to Hermite Polynomials at infinity.

\begin{itemize}
\item \textbf{Recurrence relation}
	\begin{equation}
		\begin{cases}
		\tilde{H}_0(x) = e^{-x^2/2}\\
		\tilde{H}_1(x) = \sqrt{2} x e^{-x^2/2}\\
		\tilde{H}_{n+1}(x) = x \sqrt{\frac{2}{n+1}} \tilde{H}_n(x) - \sqrt{\frac{n}{n+1}} \tilde{H}_{n-1}(x), \qquad n \geq 1
		\end{cases}
	\end{equation}
\item \textbf{Derivatives}\\
	The recursion relation for the $k$-th derivative of the function of order $n$ is:
	\begin{equation}
		\tilde{H}_n^{(k)}(x) = \sqrt{\frac{n}{2}}\tilde{H}_{n-1}^{(k-1)}(x) - \sqrt{\frac{n+1}{2}}\tilde{H}_{n+1}^{(k-1)}(x)
	\end{equation}
	Using this recursion formula we end up having an expression involving only Hermite Functions $ \tilde{H}_n^{(0)}(x) $, that can be computed using the recurrence relation, and derivatives of the first Hermite Function $ \tilde{H}_0^{(k)} $ that have the following form:
	\begin{equation}
		\tilde{H}_0^{(k)} = a_0 e^{-x^2/2} + a_1 x e^{-x^2/2} + a_2 x^2 e^{-x^2/2} + \ldots + a_k x^k e^{-x^2/2} 
	\end{equation}
	The values $ \left\lbrace a_i \right\rbrace_{i=0}^k $ can be found using the following table:
	\begin{center}
	\begin{tabular}{c|c|c|c|c|c|c|c|c|c|c}
		$k$ & $a_0$ & $a_1$ & $a_2$ & $a_3$ & $a_4$ & $a_5$ & $a_6$ & $a_7$ & $a_8$ & $\ldots$ \\ \hline
		0 & 1 & & & & & & & & & $\ldots$ \\ \hline
		1 & & -1 & & & & & & & & $\ldots$ \\ \hline
		2 & -1 & & 1 & & & & & & & $\ldots$ \\ \hline
		3 & & 3 & & -1 & & & & & & $\ldots$ \\ \hline
		4 & 3 & & -6 & & 1 & & & & & $\ldots$ \\ \hline
		5 & & -15 & & 10 & & -1 & & & & $\ldots$ \\ \hline
		6 & -15 & & 45 & & -15 & & 1 & & & $\ldots$ \\ \hline
		7 & & 105 & & -105 & & 21 & & -1 & & $\ldots$ \\ \hline
		8 & 105 & & -420 & & 210 & & -28 & & 1 & $\ldots$ \\ \hline
		$\vdots$ & $\vdots$ & $\vdots$ & $\vdots$ & $\vdots$ & $\vdots$ & $\vdots$ & $\vdots$ & $\vdots$ & $\vdots$ & 
	\end{tabular}
	\end{center}
	that can be generated iteratively using the following rules:
	\begin{equation*}
		\begin{cases}
		A(0,0) = 1\\
		A(i,j) = 0 \qquad & \text{if $ i<j $}\\
		A(i,j) = A(i,j) - A(i-1,j-1) \qquad & \text{if $ j \neq 0 $}\\
		A(i,j) = A(i,j) + A(i-1,j+1) (j+1) \qquad & \text{if $i > j$}
		\end{cases}
	\end{equation*}
	\item \textbf{Orthogonality}\\
		\begin{align}
			w(x) &= 1 \\
			\gamma_n &= \sqrt{\pi}
		\end{align}
	\item \textbf{Gauss Quadrature points and weights}
		The Gauss points $\lbrace \tilde{x}_j \rbrace_{j=0}^N$ corresponding to $\tilde{H}_{N+1}(x)$ can be obtained using the Golub-Welsh algorithm \cite{press_numerical_2007} where:
		\begin{equation}
			a_j = 0 \qquad b_j = \frac{j}{2}
		\end{equation}
		These points are exactly the same of the Hermite Polynomials in \eqref{eq:OrthPoly:HermitePolyPhy:GQx}.\\
		The Gauss weights are obtained by:
		\begin{equation}
			\tilde{w}_j = \frac{\gamma_N}{(N+1)\tilde{H}_N^2(x_j)}
		\end{equation}
\end{itemize}

\subsubsection{Hermite Probabilists' Polynomials}
The Hermite Physicists Polynomials denoted by $H_n(x)$ are eigenfunctions of the Sturm-Liouville problem:
\begin{equation}
\left( e^{-x^2} He_n'(x) \right)' + \lambda_n e^{-x^2} He_n(x) = 0 , \qquad \forall x \in I:=(-\infty,\infty) \wedge \lambda \geq 0
\end{equation}

\begin{itemize}
	\item \textbf{Recurrence relation}
		\begin{equation}
			\begin{cases}
			He_0(x) = 1 \\
			He_1(x) = x \\
			He_{n+1}(x) = xHe_n(x) - nHe_{n-1}(x)
			\end{cases}
		\end{equation}
	\item \textbf{Derivatives} 
		\begin{equation}
			\begin{cases}
			He_n^{(k)}(x) = nHe_{n-1}^{(k-1)}(x) \\
			He_n^{(0)}(x) = He_n(x) \\
			He_0^{(k)}(x) = 0 \qquad \text{for $k>0$}
			\end{cases}
		\end{equation}
	\item \textbf{Orthogonality} 
		\begin{align}
			w(x) &= \frac{1}{\sqrt{2\pi}} e^{-x^2/2} \\
			\gamma_n &= n!
		\end{align}
	\item \textbf{Gauss Quadrature points and weights}\\
		The Gauss points $\lbrace x_j \rbrace_{j=0}^N$ corresponding to $He_{N+1}(x)$ can be obtained using the Golub-Welsh algorithm \cite{press_numerical_2007} where:
		\begin{equation}
			a_j = 0 \qquad b_j = j
		\end{equation}
		The Gauss weights are obtained by:
		\begin{equation}
			w_j = \frac{\gamma_N}{(N+1)He_N^2(x_j)}
		\end{equation}
\end{itemize}

\subsection{Laguerre Polynomials}

\section{Generalized Polynomial Chaos}

\begin{exa}[Stochastic Test Equation]\label{exa:StochasticTestEquation-gPC}
\mbox{}\\
Consider the stochastic test equation
\begin{align}
	&\frac{du(t,\xi)}{dt} = -k(\xi) u(t,\xi), \qquad u(0,\xi) = u_0\label{eq:exa:StochasticTestEquation} \\
	&\xi \sim \mathcal{U}(-1,1), \qquad \rho_\xi(\xi)= \frac{1}{2}, \qquad k(\xi) = \frac{1}{2}\xi + \frac{1}{2} \notag
\end{align}
where the decay rate is uniformly distributed in $I \in [0,1]$. Let's apply \textbf{non-normalized Legendre-chaos} on the random input as well as on the function $u(t,\xi)$, where $\left\lbrace J_i^{(0,0)}(\xi) \right\rbrace_{i=0}^N $ are the orthogonal Legendre basis functions.
\begin{align}
	k(\xi) &\approx k_N(\xi) = \sum_{i=0}^N \hat{k}_i J_i^{(0,0)} \\
		&\hat{k}_i = \frac{1}{\gamma_i} \int_{-1}^{1} k(\xi)J_i^{(0,0)}(\xi)w(\xi)d\xi \approx A^T k(\underline{\xi}) \\
	u(t,\xi) &\approx u_N(t,\xi) = \sum_{i=0}^N \hat{u}_i(t) J_i^{(0,0)} \\
		&\hat{u}_i(t) = \frac{1}{\gamma_i} \int_{-1}^{1} u(t,\xi)J_i^{(0,0)}(\xi)w(\xi)d\xi \approx A^T u(t,\underline{\xi})
\end{align}
where $ A_{j,i} = \frac{J_i^{(0,0)}(\xi_j)}{\gamma_i} w_j $ and $\left\lbrace \xi_i, w_i \right\rbrace_{i=0}^N $ is a set of quadrature points. The \textbf{gPC-expansion} of \eqref{eq:exa:StochasticTestEquation} is given by:
\begin{align}
	&\Exp\left[ \frac{du(t,\xi)}{dt} J_k^{(0,0)}(\xi) \right]_{\rho_\xi(\xi)} = \Exp\left[ -k(\xi)u(t,\xi)J_k^{(0,0)}(\xi) \right]_{\rho_\xi(\xi)}\\
	&\int_{-1}^1 \sum_{i=0}^N \frac{d\hat{u}_i(t,\xi)}{dt} J_i^{(0,0)}(\xi) J_k^{(0,0)}(\xi) \rho_\xi(\xi)d\xi = \\ &-\int_{-1}^1 \sum_{i,j=0}^N \hat{k}_i \hat{u}_j(t) J_i^{(0,0)}(\xi) J_j^{(0,0)}(\xi) J_k^{(0,0)}(\xi) \rho_\xi(\xi) d\xi \notag \\
	& \frac{d\hat{u}_k(t)}{dt} = -\frac{1}{\gamma_k} \sum_{i,j=0}^N \hat{k}_i \hat{u}_j(t) \underbrace{\int_{-1}^1 J_i^{(0,0)}(\xi) J_j^{(0,0)}(\xi) J_k^{(0,0)}(\xi) w(\xi) d\xi}_{\approx \sum_l^{2N} J_i^{(0,0)}(\xi_l) J_j^{(0,0)}(\xi_l) J_k^{(0,0)}(\xi_l) w_l = e_{ijk}}
\end{align}
The \textbf{initial conditions} for the gPC-expansion are:
\begin{equation}
	\hat{u}_i(0) = \frac{1}{\gamma_i} \int_{-1}^1 u(0,\xi) J_i^{(0,0)}(\xi) w(\xi) d\xi = 
	\begin{cases}
	u_0 \qquad & i=0\\
	0 \qquad & i \geq 1
	\end{cases}
\end{equation}
The \textbf{expectation} of the solution is given by
\begin{align}
	\Exp\left[ u(t,\xi) \right]_{\rho_\xi} &= \int_{-1}^1 u(t,\xi) \rho_\xi(\xi) d\xi \approx \int_{-1}^1 \sum_{i=0}^N \hat{u}_i(t) J_i^{(0,0)}(\xi) \rho_\xi(\xi) d\xi \\
	&= \sum_{i=0}^N \hat{u}_i(t) \frac{1}{2} \int_{-1}^1 J_i^{(0,0)}(\xi) w(\xi) d\xi = \hat{u}_0(t) \notag
\end{align}
The \textbf{variance} of the solution is given by
\begin{align}
	\Var [u(t,\xi)]_{\rho_\xi(\xi)} &= \Exp \left[ \left( u(t,\xi) - \mu_u(t) \right)^2 \right]_{\rho_\xi(\xi)} \\
	&\approx \Exp [ u_N^2(t,\xi) ]_{\rho_\xi(\xi)} - 2 \mu_u(t) \Exp [u_N(t,\xi)]_{\rho_\xi(\xi)} + \mu_u^2(t) \notag \\
	& = \int_{-1}^1 \sum_{i,j=0}^N \hat{u}_i(t) \hat{u}_j(t) J_i^{(0,0)}(\xi) J_j^{(0,0)}(\xi) \rho_\xi(\xi) d\xi - \mu_u^2(t) \notag \\
	& = \frac{1}{2} \sum_{i=0}^N \hat{u}_i^2(t) \gamma_i - \hat{u}_0^2(t) = \frac{1}{2} \sum_{i=1}^N \hat{u}_i^2(t) \gamma_i \notag
\end{align}
If \textbf{normalized Legendre-chaos} is employed, then some modifications to the equations have to be considered. The normalized basis are given by $ \tilde{J}_i^{(0,0)}(\xi) = \frac{J_i^{(0,0)}(\xi)}{\sqrt{\gamma_i}} $, thus
\begin{align}
	\hat{k}_i = \int_{-1}^{1} k(\xi)\tilde{J}_i^{(0,0)}(\xi)w(\xi)d\xi \approx A^T k(\underline{\xi}) \label{eq:gPC-kExpansion} \\
	\hat{u}_i(t) = \int_{-1}^{1} u(t,\xi)\tilde{J}_i^{(0,0)}(\xi)w(\xi)d\xi \approx A^T u(t,\underline{\xi})
\end{align}
where $ A_{j,i} = \tilde{J}_i^{(0,0)}(\xi_j) w_j $ and $\left\lbrace \xi_i, w_i \right\rbrace_{i=0}^N $ is a set of quadrature points. The \textbf{gPC-expansion} is then written as
\begin{equation}
	\frac{d\hat{u}_k(t)}{dt} = - \sum_{i,j=0}^N \hat{k}_i \hat{u}_j(t) \underbrace{\int_{-1}^1 \tilde{J}_i^{(0,0)}(\xi) \tilde{J}_j^{(0,0)}(\xi) \tilde{J}_k^{(0,0)}(\xi) w(\xi) d\xi}_{\approx \sum_l^{2N} \tilde{J}_i^{(0,0)}(\xi_l) \tilde{J}_j^{(0,0)}(\xi_l) \tilde{J}_k^{(0,0)}(\xi_l) w_l = e_{ijk}}
\end{equation}
The \textbf{initial conditions} are given by
\begin{equation}
	\hat{u}_i(0) = \int_{-1}^1 u(0,\xi) \tilde{J}_i^{(0,0)}(\xi) w(\xi) d\xi = 
	\begin{cases}
	2 \frac{u_0}{\sqrt{\gamma_0}} \qquad & i=0\\
	0 \qquad & i \geq 1
	\end{cases}
\end{equation}
The \textbf{expectation} and the \textbf{variance} of the solution for the normalized gPC are given by
\begin{align}
	\Exp\left[ u(t,\xi) \right]_{\rho_\xi} &= \int_{-1}^1 u(t,\xi) \rho_\xi(\xi) d\xi \approx \int_{-1}^1 \sum_{i=0}^N \hat{u}_i(t) \tilde{J}_i^{(0,0)}(\xi) \rho_\xi(\xi) d\xi \\
	&= \sum_{i=0}^N \frac{\hat{u}_i(t)}{\sqrt{\gamma_i}} \frac{1}{2} \int_{-1}^1 J_i^{(0,0)}(\xi) w(\xi) d\xi = \frac{\hat{u}_0(t)}{\sqrt{\gamma_0}} \notag \\
	\Var [u(t,\xi)]_{\rho_\xi(\xi)} &= \Exp \left[ \left( u(t,\xi) - \mu_u(t) \right)^2 \right]_{\rho_\xi(\xi)} \\
	&\approx \Exp [ u_N^2(t,\xi) ]_{\rho_\xi(\xi)} - 2 \mu_u(t) \Exp [u_N(t,\xi)]_{\rho_\xi(\xi)} + \mu_u^2(t) \notag \\
	& = \int_{-1}^1 \sum_{i,j=0}^N \hat{u}_i(t) \hat{u}_j(t) \tilde{J}_i^{(0,0)}(\xi) \tilde{J}_j^{(0,0)}(\xi) \rho_\xi(\xi) d\xi - \mu_u^2(t) \notag \\
	& = \frac{1}{2} \sum_{i=0}^N \hat{u}_i^2(t) - \hat{u}_0^2(t) = \frac{1}{2} \sum_{i=1}^N \hat{u}_i^2(t) \notag
\end{align}
An advantage of using orthonormal polynomials is that we don't need to compute $ \left\lbrace \gamma_i \right\rbrace_{i=1}^N $ values that involve factorials and can become hard to compute accurately for big $N$ values.
\end{exa}

\subsection[Time Dependent gPC]{Time Dependent generalized Polynomial Chaos}
In order to improve the performances on time dependent ODEs, one can employ Time Dependent generalized Polynomial Chaos (TDgPC) first introduced in \cite{gerritsma_time-dependent_2010}.

Stop criteria
\begin{equation}\label{eq:TDgPC-StopCrit}
	\max\left( \vert \fourIdx{}{j}{}{2}{\hat{u}}(t) \vert, \dots, \vert \fourIdx{}{j}{}{N}{\hat{u}}(t) \vert \right) \geq \frac{\vert \fourIdx{}{j}{}{1}{\hat{u}}(t) \vert}{\theta}
\end{equation}

Integral relation
\begin{equation}\label{eq:TDgPC-IntegralRelation}
	\int_{I_{\psi_j}} g(\psi_j)f_{\psi_j}(\psi_j) d\psi_j = \int_{I_\xi} g(\mathbf{T}(\xi))f_\xi(\xi) d\xi
\end{equation}
where $ \mathbf{T}_j(\xi) $ is the composition $ T_1 \circ \dots \circ T_j $ of transformations from the variable $\xi$ to the variables $\psi_1,\dots,\psi_j$.

\begin{exa}[Stochastic Test Equation - continuing example \ref{exa:StochasticTestEquation-gPC}]
\mbox{}\\
Consider the solution $ u(t_j,\psi_{j-1}) $ at time $t_j$ that satisfy the stopping criteria \eqref{eq:TDgPC-StopCrit}. Let's define a \textbf{new random variable} $\psi_j$ corresponding to such solution:
\begin{align}
	\psi_j = \mathbf{T}_j(\xi) = u(t_j, \psi_{j-1}) &\approx \sum_{i=0}^N [\fourIdx{}{j-1}{}{i}{\hat{u}}(t)] [\fourIdx{}{j-1}{}{i}{\phi}(\psi_{j-1})] \\
	&= \sum_{i=0}^N [\fourIdx{}{j-1}{}{i}{\hat{u}}(t)][\fourIdx{}{j-1}{}{i}{\phi}(\mathbf{T}_{j-1}(\xi))] \notag
\end{align}
We now seek for the best set of orthonormal polynomials in order to describe the distribution of this random variable. We use a generalized version of \textbf{Gram-Schmidt orthogonalization} algorithm \cite{stoer_introduction_2002} for weighted normed spaces. The orthogonalization is started with the Vandermonde matrix of $ \psi $. We finally obtain a set $ \left\lbrace \fourIdx{}{j}{}{i}{\phi}(\psi_j) \right\rbrace_{i=0}^N $ of basis functions s.t.
\begin{equation}
	\int_I [\fourIdx{}{j}{}{i}{\phi}(\psi_j)] [\fourIdx{}{j}{}{k}{\phi}(\psi_j)] f_{\psi_j}(\psi_j) d\psi = \delta_{ik}, \qquad \text{for $ i,k = 0, \dots, N $}
\end{equation}
We now need to rewrite the system of ODE with respect to this basis functions. First we rewrite the \textbf{initial conditions}, that are given by:
\begin{align}
	u(t,\psi_j) &\approx \sum_{i=0}^N [\fourIdx{}{j}{}{i}{\hat{u}}(t)] [\fourIdx{}{j}{}{i}{\phi}(\psi_{j})] \label{eq:TDgPC-newExpansion} \\
	\fourIdx{}{j}{}{i}{\hat{u}}(t_j) &= \frac{1}{\Vert \fourIdx{}{j}{}{i}{\phi} \Vert}_{f_{\psi_j}} \int_{I_{\psi_j}} u(t_j,\psi_j) [\fourIdx{}{j}{}{i}{\phi}(\psi_{j})] f_{\psi_j}(\psi_j) d\psi_j \\
	&= \frac{1}{\Vert \fourIdx{}{j}{}{i}{\phi} \Vert}_{f_{\psi_j}} \underbrace{\int_{I_{\xi}} u(t_j,\mathbf{T}(\xi)) [\fourIdx{}{j}{}{i}{\phi}(\mathbf{T}(\xi))] f_{\xi}(\xi) d\xi}_{\approx \sum_{l=0}^{Q N} u(t_j,\mathbf{T}(\xi_l)) [\fourIdx{}{j}{}{i}{\phi}(\mathbf{T}(\xi_l))] w_l} \notag
\end{align}
where $ Q $ determines the precision of the quadrature rule, that is used for estimating an integral that can possibly not be a polynomial. We now \textbf{rewrite the ODE} in terms of the new polynomials. For the parameter $k(\xi)$ the expansion \eqref{eq:gPC-kExpansion} can still be used, while the new expansion \eqref{eq:TDgPC-newExpansion} will be used for $ u(t,\psi_j) $. We plug these expansion in the weak formulation of gPC, obtaining
\begin{align}
	&\Exp\left[ \frac{du(t,\psi_j)}{dt} \fourIdx{}{j}{}{i}{\phi}(\psi_j) \right]_{f_{\psi_j}} = \Exp\left[ -k(\xi)u(t,\psi_j)[\fourIdx{}{j}{}{i}{\phi}(\psi_j)] \right]_{f_{\psi_j}}\\
	&\frac{d[\fourIdx{}{j}{}{i}{\hat{u}}(t)]}{dt} = - \frac{1}{\Vert \fourIdx{}{j}{}{i}{\phi} \Vert}_{f_{\psi_j}} \sum_{l,i=0}^N \hat{k}_l [\fourIdx{}{j}{}{i}{\hat{u}}(t)] \underbrace{\int_{-1}^1 \tilde{J}_l^{(0,0)}(\xi) [\fourIdx{}{j}{}{i}{\phi}(\mathbf{T}(\xi))] [\fourIdx{}{j}{}{k}{\phi}(\mathbf{T}(\xi))] f_\xi(\xi) d\xi}_{\approx \sum_{n=0}^{Q N} \tilde{J}_l^{(0,0)}(\xi_n) [\fourIdx{}{j}{}{i}{\phi}(\mathbf{T}(\xi_n))] [\fourIdx{}{j}{}{k}{\phi}(\mathbf{T}(\xi_n))] w_n = e_{lik}}
\end{align}
The \textbf{mean} and the \textbf{variance} can now be computed using
\begin{align}
	\Exp \left[ u(t,\psi_j) \right]_{f_{\psi_j}} &= \sum_{i=0}^N \fourIdx{}{j}{}{i}{\hat{u}}(t) \underbrace{\int_{-1}^{1} [\fourIdx{}{j}{}{i}{\phi}(\mathbf{T}(\xi))] f_\xi(\xi) d\xi}_{\approx \sum_{n=0}^{QN} [\fourIdx{}{j}{}{i}{\phi}(\mathbf{T}(\xi_n))] w_n} \\
	\Var \left[ u(t,\psi_j) \right]_{f_{\psi_j}} &= \left[\sum_{i=0}^N \fourIdx{}{j}{}{i}{\hat{u}}(t) \underbrace{\int_{-1}^{1} [\fourIdx{}{j}{}{i}{\phi}(\mathbf{T}(\xi))] [\fourIdx{}{j}{}{i}{\phi}(\mathbf{T}(\xi))] f_\xi(\xi) d\xi}_{\approx \sum_{n=0}^{QN} [\fourIdx{}{j}{}{i}{\phi}(\mathbf{T}(\xi_n))] [\fourIdx{}{j}{}{i}{\phi}(\mathbf{T}(\xi_n))] w_n} \right] - \mu_u^2(t)
\end{align}
\end{exa}

\section{Probabilistic Collocation Method}
Probabilitsic collocation is a non-intrusive approach used to solve stochastic problems.\\
Let 

%%%%%%%%%%%%%%%%%%%%%%%%%%%%%%%%%%%%%%%%%%%%%%%%%%%%%%%%%%%%%
%% APPENDICES
%%%%%%%%%%%%%%%%%%%%%%%%%%%%%%%%%%%%%%%%%%%%%%%%%%%%%%%%%%%%%
\appendix

%% ==> Write your text here or include other files.

%\input{FileName} %You need a file 'FileName.tex' for this.

%%%%%%%%%%%%%%%%%%%%%%%%%%%%%%%%%%%%%%%%%%%%%%%%%%%%%%%%%%%%%
%% BIBLIOGRAPHY AND OTHER LISTS
%%%%%%%%%%%%%%%%%%%%%%%%%%%%%%%%%%%%%%%%%%%%%%%%%%%%%%%%%%%%%
%% A small distance to the other stuff in the table of contents (toc)
\addtocontents{toc}{\protect\vspace*{\baselineskip}}

%% The Bibliography
%% ==> You need a file 'literature.bib' for this.
%% ==> You need to run BibTeX for this (Project | Properties... | Uses BibTeX)
%\addcontentsline{toc}{chapter}{Bibliography} %'Bibliography' into toc
%\nocite{*} %Even non-cited BibTeX-Entries will be shown.
\bibliographystyle{acm} %Style of Bibliography: plain / apalike / amsalpha / ...
\bibliography{refs} %You need a file 'literature.bib' for this.

%% The List of Figures
%\clearpage
%\addcontentsline{toc}{chapter}{List of Figures}
%\listoffigures

%% The List of Tables
%\clearpage
%\addcontentsline{toc}{chapter}{List of Tables}
%\listoftables

\end{document}